%*******************************************************************************
%                                                                              *
%                 Datei: header.tex                                            *
%                                                                              *
%                 Stand: 10.10.2013   11.02 Uhr   (Elt)                        *
%                                                                              *
%*******************************************************************************

\documentclass[%
    paper=a4,            % Layout fuer Din A4
    %oneside,          % einseitiger Druck
	twoside,             % Layout fuer beidseitigen Druck
    fontsize=12pt,       % Schriftgroesse 12pt
    %openright,        % Kapitel dürfen nur auf einer rechten Seite beginnen
    openany,          % Kapitel dürfen rechts oder links beginnen
    halfparskip,      % eine halbe Zeile Abstand zw. Absätzen
    headsepline,         % horizontale Linie unter Kolumnentitel
    headinclude,         % Kopfzeile wird Seiten-Layouts mit beruecksichtigt
    plainheadsepline,    % horizontale Linie auch beim plain-Style
    %footsepline,      % Fußzeilenlinie
    parskip=half,        % Absatzabstand statt Absatzeinzu
    bibliography=totoc,  % Literaturverz. wird ins Inhaltsverzeichnis eingetragen
    %idxtotoc,          % Index im Inhaltsverzeichnis
    listof=totoc,        % Abbildungs und TabellenVZ ins InhaltsVZ
    ]{scrbook}


\usepackage{longtable}
\usepackage{titlesec}% Ändert die Größe der Überschriften subsubsection
%\titleformat*{\subsubsection}{\large\bfseries}

\usepackage[utf8]{inputenc}
\usepackage{titleref}
% deutsche Silbentrennung etc.
\usepackage[ngerman]{babel}      % neue Rechtschreibung

%schriftarten
\usepackage[T1]{fontenc}
\newcommand{\changefont}[3]{
\fontfamily{#1} \fontseries{#2} \fontshape{#3} \selectfont}

%abkürzungsverzeichnis
\usepackage[printonlyused, footnote, nohyperlinks]{acronym}
%\usepackage{glossaries}

%\usepackage{natbib}
\usepackage[square,sort,comma,numbers]{natbib}

%fußnote nummerierung durchgehen
\usepackage{chngcntr} 
\counterwithout{footnote}{chapter}

% Grafiken: PDF, GIF, PNG
\usepackage{graphicx}
\usepackage{subfigure}
\usepackage{pdfpages}
\usepackage{float}
\usepackage{wrapfig}


%tür tabellen
%\usepackage{booktabs}

% Farben
\usepackage{color}
\definecolor{link_color}{rgb}{0,0,0.6}
\definecolor{ListingBG}{rgb}{0.95,0.95,0.95}

% Hyperlinks (anklickbar im PDF)
\usepackage[%
    pdftitle={Sortieralgorithmus Bubblesort},%
    pdfauthor={Nick Koslowski},%
    pdfpagemode=UseOutlines
]{hyperref}   

\usepackage{blindtext}
    
\hypersetup {
     breaklinks = {true},      % Erlaubt Zeilenumbrüche in Links
     colorlinks = {true},      % Benutze farbige Links
     citecolor = {link_color}, % Farbe für Zitate
     linkcolor = {link_color}, % beeinflusst Inhaltsverz. und Seitenzahlen
     urlcolor = {link_color},  % Weblink-Farbe
     pdftitle = {Sortieralgorithmus Bubblesort}, % Titel der Arbeit
     pdfsubject = {Unterweisungsentwurf}, % Thema der Arbeit
     pdfauthor = {Nick Koslowski},  % AutorIN der Arbeit
     pdfkeywords = {Hier ein paar Stichwörter}, % Stichwörter zur Arbeit
     pdfproducer = {pdfLaTeX}, % Erzeugt durch
     pdfcreator = {MacTeX},    % Erstellt mit
     pdfstartview = {FitV},
     pdfview = {FitH},
     pdffitwindow = {true}
}


% Quellcode Formatierung
\usepackage{listings}
\lstset{%
    language=Python, % Programmiersprache
    numbers=left, % Zeilennummern
    stepnumber=1, % jede Zeile
    numbersep=5pt, % Abstand zum Quellcode
    numberstyle=\tiny, % Schriftgröße
    breaklines=true, % Zeilenumbrüche zulassen
    breakautoindent=true, % Einrücken nach Umbruch
    tabsize=2,  % Tabulator
    basicstyle=\footnotesize, %
    showspaces=false, % Leerzeichen anzeigen (true -> underscore)
    showstringspaces=false, % Leerzeichen in Strings
    backgroundcolor=\color{ListingBG}, % Hintergrundfarbe
    captionpos=b,   % Position der Beschreibung (b: bottom)
    %keywordstyle=\color{red}\bfseries
}

\usepackage{setspace}       % Zeilenabstand einstellbar
\onehalfspacing             % eineinhalbzeilig einstellen
\usepackage{scrpage2}       % Kopf und Fusszeilen-Layout 


\renewcommand{\headfont}{\normalfont\sffamily}    % Kolumnentitel serifenlos
\renewcommand{\pnumfont}{\normalfont\sffamily}    % Seitennummern serifenlos
\pagestyle{scrheadings}
\ihead[]{\headmark}              % Kolumnentitel immer oben innen
\ohead[\pagemark]{\pagemark}     % Seitennummern immer oben aussen
\ofoot[]{}                       % Seitennummern in der Fusszeile loeschen

\renewcommand{\bibname}{Literatur} % Literaturverzeichnis wird zu Literatur
\renewcommand{\figurename}{Abb.}   % Abbildung wird zu Bild

% erweiterte Tabellen
\usepackage{array}

% Formelsatz
\usepackage{amsmath}

% Definition eigener Operatoren (im Header)
\DeclareMathOperator{\rg}{Rang}  

% Fortlaufende Kapitelüberschriften in der Kopfzeile
%\pagestyle{headings}

% Stil des Literaturverzeichnis
\bibliographystyle{alphadin}

%text im Literaturverzeichnis
\setbibpreamble{Auflistung aller Quellen, die zum erstellen dieses Dokumentes verwendet wurden.\par\bigskip}
